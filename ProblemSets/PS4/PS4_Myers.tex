\documentclass[12pt]{article}
\usepackage[margin=1in]{geometry}
\usepackage{hyperref}

\begin{document}
\begin{flushright}
\textbf{\Large Problem Set 4} \\
Brent Myers \\
02/24/26
\end{flushright}

\section{Question 5: JSON Exercise}

In this exercise, I downloaded a JSON file containing historical events from a web API using \texttt{wget} within an R script. I then converted the JSON data into an R data frame using the \texttt{jsonlite} and \texttt{tidyverse} packages.

\subsection{Part (d): Object Types}
\begin{itemize}
    \item \texttt{class(mydf)} returned: \texttt{"tbl\_df" "tbl" "data.frame"}. This means \texttt{mydf} is a tibble, which is the tidyverse version of a data frame.
    \item \texttt{class(mydf\$date)} returned: \texttt{"character"}. This means the date column is stored as a character string, not as a date object.
\end{itemize}

\subsection{Part (e): First 6 Rows}
The first 6 rows of the data frame show historical events from the year 1 AD, including events from the Roman Empire and Asia. The data frame contains 6 columns: \texttt{date}, \texttt{description}, \texttt{lang}, \texttt{category1}, \texttt{category2}, and \texttt{granularity}.

\begin{verbatim}
# A tibble: 6 x 6
  date  description                        lang  category1 category2 granularity
  <chr> <chr>                              <chr> <chr>     <chr>     <chr>
1 1     Tiberius, under order of Augustus… en    By place  Roman Em… year
2 1     Gaius Caesar and Lucius Aemilius … en    By place  Roman Em… year
3 1     Gaius Caesar marries Livilla, dau… en    By place  Roman Em… year
4 1     Quirinius becomes a chief advisor… en    By place  Roman Em… year
5 1     Areius Paianeius becomes Archon o… en    By place  Roman Em… year
6 1     The ''Yuanshi'' era of the Chines… en    By place  Asia      year
\end{verbatim}

\section{Question 6: sparklyr Exercise - (Did Not Complete)}

I was \textbf{\underline{unable}} to run the sparklyr exercise because the \texttt{sparklyr} package could not be installed on OSCER. The installation failed due to dependency version conflicts---specifically, the version of \texttt{dplyr} available on OSCER (0.8.5) was too old for the required \texttt{dbplyr} package (which requires dplyr $\geq$ 1.1.2). The R script \texttt{PS4b\_Myers.R} contains all of the commands that would have been executed had the package been available.

Based on the exercise instructions, the expected answers would be:
\begin{itemize}
    \item \textbf{Step 7:} \texttt{df1} would be of class \texttt{tbl\_df} (a local tibble), while \texttt{df} would be of class \texttt{tbl\_spark} (a Spark DataFrame).
    \item \textbf{Step 8:} The column names in the Spark DataFrame use underscores instead of periods (e.g., \texttt{Sepal\_Length} instead of \texttt{Sepal.Length}), because Spark does not allow periods in column names.
\end{itemize}

\section{Data Sources of Interest}

There are several data sources I would be interested in scraping for research purposes:

\begin{enumerate}
    \item \textbf{SEC EDGAR:} The SEC's EDGAR database provides free access to corporate filings, including 10-K and 10-Q reports. These filings contain financial data, footnotes, and disclosures that are useful for empirical accounting research. The SEC provides an API for programmatic access.
    
    \item \textbf{FASB Accounting Standards:} The Financial Accounting Standards Board publishes accounting standards (ASCs and ASUs) that govern financial reporting. Scraping and analyzing the text of these standards could support research on the rules-based versus principles-based nature of accounting regulation.
    
    \item \textbf{Federal Reserve Economic Data (FRED):} The St. Louis Fed's FRED database offers over 800,000 economic time series, accessible through a free API for R and Python. This data is useful for macroeconomic and financial research.
\end{enumerate}

\end{document}
